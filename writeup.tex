\documentclass{article}
\usepackage{graphicx} % Required for inserting images
\usepackage{amsmath}

\title{Luck vs Skill in College Basketball}
\author{Jake Browning, Rohan Mawalkar, Seth Peacock}
\date{November 2024}

\begin{document}

\maketitle

\section{Letter to the Editor}
In addition to this report, write a one-page letter to the newspaper chief editor, explaining the main results of
the report and suggesting findings that can be communicated with the basketball fans reading the newspaper.

\section{Overview}
Describe the problem, your model, results and how your model performed.

\section{Introduction}
Rephrase the problem
Problems are open-ended and there are many ways to interpret and address them. Explain how you approached the problem.

\section{Methods}
Explain your model
Clearly state and justify ALL assumptions your model uses
Motivate your model. Why did you choose your approach?
Clearly describe your model
Clearly define all variables
Include tables and figures to make it easier to understand
Analyze your model
What are the strengths and weaknesses of your model?
How could you test your model? How stable are the results to noise? 
If you had more time, how would you expand/improve your model?

\subsection{Our ``Elo'' Ranking}
Traditionally, a player's Elo ranking is updated after each game based on the following formula:

\[
\]


The Elo score after a team's $n+1^{\text{th}}$ game is given by

\[
\gamma_{n+1} = \gamma_n + \alpha\frac{\gamma^{opp}_n}{\gamma_n}\frac{S_{\text{winner}}}{S_{\text{loser}}}
\]
\[
\gamma_{n+1} = \gamma_n - \alpha\frac{\gamma_n}{\gamma^{opp}_n}\frac{S_{\text{winner}}}{S_{\text{loser}}}
\]

where $\gamma_n$ is the team's current Elo score (after week $n$), $\gamma^{opp}_n$ is the opposing team's current Elo score, $\alpha > 0$ is a weight parameter, and $S_{\text{winner}}$, $S_{\text{loser}}$ are the scores of the winner/loser.
\subsection{Weaknesses}

\subsection{Rejected Ideas}

\section{Results}
What does your model say about the question you have been given?

\section{Next Steps}
When updating our ``Elo'' scores, we had an artbitrary parameter $\alpha$. If we had more time, we would have liked to rerun the experiment with a range of $\alpha$ and see if we obtained similar results.

\bibliographystyle{plain}
\bibliography{bibliography}

\end{document}
